\chapter{Literature Review}

This chapter reviews the currently available implementations of various
components in the processing pipeline. This chapter would be divided into
two sections, each dealing with a process --- the first section would
describe transcription, and the second section captioning.

\section{Transcription}

Figure \ref{trans} provided the transcription pipeline of three components
--- a resampler, a diarizer and a transcription engine. We would go into
each component in detail.

Furthermore, to enable transcription of multi-channel recording, the concept
of voice-activity detection would also be discussed.

\subsection{Resampling}

In the context of this project, resampling is concerned with converting an
audio stream from different specifications to a standard one, before passing to
other processing steps. Some of the common attributes of an audio stream are:

\begin{table}[H]
    \begin{tabu}{X[l]X[3]X[2]}
        \textbf{Attribute} & \textbf{Definition} & \textbf{Examples} \\
        \midrule
        Sample rate &
        (defined for PCM audio)\newline Number of audio samples per
        second &
        CD audio: 44100 Hz \\

        Bit depth &
        (defined for PCM audio)\newline Number of bits to represent
        a sample &
        CD audio: 16 bits \\
        
        Number of channels & & \\
        
        Format (codec) & &        
    \end{tabu}
    \caption{Common audio stream attributes}
\end{table}


It comprises of three tasks\footnote{
The canonical definition of resampling is only concerned with the first task.}:

\begin{itemize}
    \item Sample rate conversion --- changing the sample rate of the audio
    \item 
\end{itemize}


\section{Captioning}