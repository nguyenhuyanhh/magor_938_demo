\chapter{Literature Review}

This chapter reviews the currently available implementations of various
components in the processing pipeline. This chapter would be divided into
two sections, each dealing with a process --- the first section would
describe transcription, and the second section captioning.

Most of the implementations chosen are free, open-source and cross-platform.

\section{Transcription}

Figure~\ref{trans} provided the transcription pipeline of three components
--- a resampler, a diarizer and a transcription engine. We would go into
each component in detail.

Furthermore, to enable transcription of multi-channel recording, the concept
of voice-activity detection would also be discussed.

\subsection{Resampling}

In the context of this project, resampling is concerned with converting an
audio stream from different specifications to a standard one, before passing to
other processing steps. Some common attributes of an audio stream are:

\begin{longtabu}{X[1.5,l]X[4]X[2.5]}
    \textbf{Attribute} & \textbf{Description} & \textbf{Examples} \\
    \midrule
    \endhead{}
    Sample rate &
    (defined for PCM audio)\newline Number of audio samples per
    second~\cite{weik1995communications} &
    CD audio: 44100 Hz~\cite{sample-rate} \\
    Bit depth &
    (defined for PCM audio)\newline Number of bits to represent
    a sample~\cite{thompson2005understanding} &
    CD audio: 16 bits~\cite{iec60908} \\
    Number of channels &
    Number of independent audio channels (to create a
    perception of depth) &
    Mono (1-channel)/\newline
    Stereo (2-channel)\cite{mono-stereo} \\
    Audio coding format &
    The specific encoder/ decoder used to create the audio stream;
    usually associated with a certain file extension &
    MPEG-2 Audio Layer-III (\texttt{.mp3})~\cite{mp3}\newline
    WAVE (\texttt{.wav})~\cite{wav} \\
    \caption{Common audio stream attributes}
\end{longtabu}

Respectively, in order to resample audio streams, the following tasks
are performed on the original audio stream\footnote{The canonical
definition of resampling is only concerned with the first task
(sample rate conversion).}:

\begin{itemize}
    \item Sample rate conversion --- changing the sample rate of the audio
    (for instance, from 44100Hz to 16000Hz)
    \item Sample format conversion --- changing the type of the sample
    (for instance, from 16-bit to 8-bit samples)
    \item Channel rematrixing --- changing the number of channels
    (for instance, from stereo to mono audio)
    \item Transcoding --- changing the audio coding format (for instance,
    from WAVE to MPEG Layer-3)
\end{itemize}

There are two software packages to perform all the above tasks:

\subsubsection{Sound eXchange --- SoX (\texttt{sox})}

SoX is a cross-platform command line utility that supports conversion
between a wide range of audio formats. Additionally, the utility could
apply effects, and play and record audio files~\cite{sox-docs}.

The latest version of SoX, 14.4.2 was released in February
2015~\cite{sox-cl}. The status of development work is uncertain.

SoX is written in C, and there are a number of wrappers to other
programming languages.

\subsubsection{FFmpeg (\texttt{ffmpeg})}

FFmpeg is another cross-platform utility to record, convert and stream
audio and video~\cite{ffmpeg}.

The latest stable version of FFmpeg, 3.4 was released in October
2017~\cite{ffmpeg-dl}. The library is still actively developed.

\section{Captioning}