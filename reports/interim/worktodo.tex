\chapter{Summary of Work to be Done}

This chapter presents a summary of the remaining work items for the project.
Section 3.1 outlines the effort to extend the implemented transcription system
to handle multi-channel recordings, which is a major application focus. Section
3.2 details the necessary system improvements that needs to be implemented to
realise all the objectives in Chapter 1\footnote{See Section 1.2 Objective.}.

\section{Modules and Procedures for Multi-channel Recording Transcription}

A major application focus in SLRG is multi-channel recording transcription,
which is applied in a meeting room scenario. In this scenario, there are a number
of speakers (denote \texttt{n}) participating in a meeting, having natural
conversation with each other. Each speaker is speaking into his/ her own
close-talk microphone, and there is a fix number of far-talk microphones
(denote \texttt{m}) to record the entire meeting room. Despite being close
to the speaker, in most meeting room environments the close-talk microphone
would also capture other speakers' voice and background noise, which could
impede transcription. The problem statement is, given the set of \texttt{n}
close-talk and \texttt{m} far-talk recordings, perform an accurate speaker
identification, and based on that derive an accurate transcription for the
meeting.

The proposed workflow comprises of an initial stage, using the process of
voice activity detection (VAD) across all the close- and far-talk recordings
to determine which segments of the recordings are spoken by whom. All the
remaining false segments will be clipped to produce \texttt{n} (hopefully pure)
close-talk recordings which will be transcribed. The implementation would
involve developing the VAD module and procedure.

\section{System-wide Improvements}

As of this moment, the first three qualitative objectives of the project have
been relatively satisfied --- Modularity, Extensibility and Robustness. Effort
would be focused in realising the two remaining objectives, which is Versioning
and Logging and Reporting.

\subsection{Versioning}

The final system must be able to accommodate different module versions, and for
each module version the output of such module must also be versioned. The proposed
method to achieve this goal is to introduce the concept of \texttt{process}, a
configuration of procedures and modules; when a module is upgraded, the associated
procedure would be forked into a different procedure containing the module, and
the initial process would be forked into a different process containing the new
procedure. Processes should have separate folders under \texttt{data/}, resulting
in this folder structure:

\begin{lstlisting}
    data/
        process_id_1/
            file_id_1/
                raw/
                module_1/
                module_2/
                ...
                module_n/
                temp/
            file_id_2/
                ...
            ...
        process_id_2/
            file_id_1/
            file_id_2/
\end{lstlisting}

The folder structure under \texttt{file\_id} remains the same, which allows for
efficient evaluation of different version outputs. To specify the processes in
system manifest, the following structure is proposed:

\begin{lstlisting}
    {
        "processes": {
            "process_id_1": {
                "procedure_id_1": [],
                "procedure_id_2": [],
                ...
            },
            "process_id_2": {
                "procedure_id_1": [],
                "procedure_id_2": [],
                ...
            },
            ...
        },
        ...
    }
\end{lstlisting}

To implement this structure, the system executable must be re-engineered to load
the new manifest and output data to the data folders correctly. There should be
minimal changes to the implemented modules, as there are no hard-coded paths that
could be affected by the change in data folder structure.

\subsection{Logging and Reporting}

The final system must exhibit developer-friendly behaviours --- it must have
adequate debugging logs, and it must provide an interface to report common
statistics about the current data.

For logging, the system executable and modules are all using the Python standard
library module \texttt{logging} to produce command-line logs, in the following
format:

\begin{lstlisting}
    %(asctime)s (%(name)s | %(levelname)s) : %(message)s
\end{lstlisting}

For reporting, the proposed addition to the system executable is a reporting
function to aggregate data from the data folders and provide a convenient reporting
format. The usual statistics to provide are the lengths of the audio/ video streams,
total processing time separated by process/ procedure, etc. Helper functionality
would be incorporated accordingly to enable reporting.